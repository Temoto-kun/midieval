\documentclass[11pt, titlepage]{book}

\usepackage{musixtex}
\usepackage{amsfonts}
\usepackage{mathptmx}
\usepackage[scaled]{helvet}

\renewcommand*{\familydefault}{\sfdefault}

\title{Trends of Music through Analysis and Comparison of Musical Structures}
\author{Crisostomo, Allan V., Doydora, Jonathan L., Bragais, Alvin, Legarte, Kennedy}
\begin{document}
    \maketitle
    
    \tableofcontents
    
    \chapter{The Problem and Its Background}
    
    This chapter presents the problem along with its background, the research assumptions for the system and the study, the conceptual frameworks of the system and the study, the scope, the limitations and the significance. At the end of this chapter, the definition of conceptual at operational terms is also presented.
    
    \section{Introduction}
    Music, like any other art form, evolves with time. Ever since the dawn of civilization, man had been producing music utilizing the influence of nascent cultures everywhere, the collective experience of composers, and the power of artistic self-expression.
    
%    Through man’s works, musical patterns emerge from the time period where they reside, and were reused and altered through the ages.
%    The explosion of musical forms in the late 19th century showed how music was able to change.
    
    \section{Background of the Study}
    
    \section{Conceptual Framework}
    
    \section{Statement of the Problem}
    
    \section{Scope and Limitations}
    
    \subsection{System}
    
    \subsection{Study}
    
    \section{Significance of the Study}
    
    \section{Definition of Terms}
    
    \section{Conceptual Terms}
    
    \section{Operational Terms}
    
    \chapter{Review of Related Literature}
    
    \section{Related Literature}
    
    \section{Related Studies}
    
    \section{Synthesis of the Study}
    
    \chapter{Research Methodology}
    
    \section{System Architecture}
    
    \begin{music}\nostartrule
        \parindent12mm
        \instrumentnumber{1}       
        \setname1{Piano}           
        \setstaffs1{2}
        \setclef1\bass             
        \generalmeter{\meterfrac{12}{8}}
        \generalsignature{-3}
        \startextract              
            \Notes\ibl0I3\qb0{I}\zq{6}\qb0{d}\tbl0\zq{9}\zq{11}\qb0{f}
                | \qlp m
            \en
            \Notes\ibl0I3\qb0{I}\zq{6}\qb0{d}
                | \ql n
            \en
            \Notes\tbl0\zq{9}\zq{11}\qb0{f}
                | \cl m
            \en
            \Notes\ibl0H3\qb0{=H}\zq{6}\qb0{c}\tbl0\zq{10}\qb0{f}
                | \qlp m
            \en
            \Notes\ibl0H3\qb0{H}\zq{6}\qb0{c}\tbl0\zq{10}\qb0{f}
                | \qlp j
            \en
        \bar
            \Notes\ibl0H3\qb0{_H}\zq{5}\qb0{c}\tbl0\zq{8}\zq{10}\qb0{e}
            | \ibl0l0\qb0{ll}\tbl0\qb0{l}
            \en
        \zendextract                 
    \end{music}
    
    \[ {\iota_\nu}_p(\nu_1,\nu_2,p) = W_p(1) - W_p(\nu_2 - \nu_1) \]
    
    Where $ {\iota_\nu}_p $ is an interval harmonicity function 
    
    \chapter{Analysis and Results}
    
    \chapter{Conclusion}
    
    \chapter*{Appendix}

\end{document}